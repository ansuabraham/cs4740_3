\documentclass{article}
\usepackage{url}
\title{Word Sense Disambiguation\\
\small{Leveraging the NLTK Toolkit and External Tools}}
\author{
Alec Story
Ansu Abraham
Craig Frey\\
Dustin Tiedemann
Michael Zhu
Thomas Levine
Whitney Foster\\
}

\begin{document}
\maketitle

\section{Overview}

We  chose to implement our system in Python, and chose to use the machine
learning algorithms in the NLTK toolkit which is also Python-based.

We wrote our main control program, classify.py, so that it can choose either
[we're just doing nb and decision trees now right?] classifier and any
combination of  features. On top of this we wrote bash scripts that automated
looping through every combination of classifiers and features, along with
porting the results directly into the scorer.  Classify.py includes a help
switch -h, that explains the switches that control the different classifiers
and features along with any additional parameters that a certain feature might
support (for example, collocation has a window size parameter which defaults to
0) Performance was excellent for most of the features, allowing us to typically
run all of the words in the test set through every combination a classifier and
feature set in under an hour. (The number of runs can get quite high when you
factor in several parameters per feature, not just on/off)

A baseline was run by choosing the most frequent sense for each word.

\subsection{Data details}

The  NLTK has a corpus reader  that imports Senseval-2 files. The Senseval-3
file format is slightly  different which necessitated a few changes. We
separated the training  and test files by word into .pos files to match  what
the existing  parser looked for. In the interest of time we modified  the
training and  test files to contain a single tagged word per chunk  (but still
allowing for multiple senses) again to match the Senseval-2  format.

\subsection{Selecting optimal feature combination}
\newcommand\ward{forward} %or maybe backwards

We used a \ward stepwise approach to indentifying optimal feature combinations.
We first ran the system with all features on, producing a total of $256=2^8$
combinations.  We then used backwards stepwise regression to identify feature
combinations with a small number of features but with high performance

\section{System}

We made heavy use of python's NLTK
package\footnote{\url{http://www.nltk.org/}}, using its classifiers as our
machine learning engines, and several of its other packages for analysis.

\subsection{Classifiers}

The NLTK was used to implement two machine learning algorithms, naive Bayes and
decision trees.  

\subsubsection{Naive Bayes}

The NLTK implements a naï bayes classifier which chooses the best sense given a
dictionary of features and assigns confidence probabilities to the chosen
sense. Naive bayes does assume an independence of features. The probability of
each feature contributes to the overall choice.

\subsubsection{Decision Tree}

The NLTK also implements a decision list classifier in which a sequence of
tests is applied to each target-word feature vector. Each test indicates a
particular sense of the word: if the test succeeds, the sense of the test is
returned otherwise the next test in the sequence is performed.

Naive Bayes supported probability estimation, while the decision tree did not.

\subsection{Confidence Cutoffs}

NLTK provides probability measures for each of its guesses when using the
maximum entropy and the naive Bayes learning methods.  We harnessed these to
determine the cutoff for making a guess, or resorting to a ``U'' if we were
unconfident in the tagging.  Occasional unknowns also occur because such
unknowns are present in the training data, and we made no effort to remove
them, so the learning algorithm views them as any other label.

% Should we mention which cutoff we used?  Which did we use?

\subsection{Bootstrapping}

We used the same probability measure from the confidence cutoffs to fuel a
bootstrapping method.  Instead of directly taking the output of the classifier,
we instead looked at its performance on the test data, and wherever it was very
confident, we copied that test instance to the training data, and repeated this
cycle for a fixed number of cycles.  Once we had enriched the training data, we
ran the classifier a final time to perform classification.

Fortunately, the classifiers were very fast, and almost all of the time the
system spends is in reading input and building feature sets, which don't need
to be repeated, so bootstrapping is very cheap to include.

% Should we mention which probability measure we used?  Which did we use?  How
% many iterations?

\section{Looking through Different Corpora}

In order to get more data to use with the bootstrapping algorithm, we attempted
to search through the Brown corpus and return all of the words that were being
disambiguated in the Senseval 3 task.  To do this we needed to find all of the
instances of the words that we needed to disambiguate and return a context
(list of words) and the position of the word in that list.  We also made sure
that we the context that we returned both started at the beginning of the
sentence and ended at the end of a sentence and had a minimum of 25 words
returned on either side of the word.  This data would then be converted into
the form of the data that we received from the Senseval document and ran
through the bootstrapping algorithm to have more data to train with.  However
when we actually tested this feature alone we found it to be very
uninformative.  First, there were very few instances of the words that we were
looking for (ex: activate only appeared 3 times), which made searching through
the entire corpus to find these words more time consuming than it was helpful.
Therefore, we decided not to use it for the actual test because it was not
informative enough by itself.

\subsection{Base Word}

Because the target words were grouped by part of speech, we included the word
as it appears literally in context as a feature.  For example, ``activate''
versus ``activating,'' which both appear under activate.v.

\subsection{Dependency Parsing}

We included as one feature the MaltParser dependency parser, and used the
engmalt.linear.mco linear support vector machine
configuration\footnote{\url{http://maltparser.org/mco/english_parser/engmalt.html}}
for parsing (the other option was an SVM with a polynomial kernel, which the
Malt website said would run just as accurately, more slowly, but with less
memory, so we decided to go with the faster option).

From the parser we rendered several features:

\begin{description}

\item[The Dependency:] whatever the first of the list of dependencies given for
the target word is.  It is difficult to represent lists in the key and value
feature format, so we did not attempt to.  Instead, we use the rest of the
features as mitigating factors.  The empty string if there is no dependency.

\item[Presence of Absence of Dependency]

\item[Number of Dependencies]

\item[The Parent:] the word with the target word as (one of) its dependencies.

\end{description}

Dependency parsing is fairly slow, requiring about a second per instance to run
on a computer running an Atom D525 processor (which is fairly underpowered, but
modern).  To help alleviate this problem, we wrote a system that saves parses
for each context, and only calls the parser if it observes a new parse, so we
could reuse parsing information across runs of our program.  Because this file
is several hundred megabytes in size, it is not included in our submitted code.
\subsection{Collocation}

Encodes the part-of-speech tag of words to the left and right of the target
word. The number of words extracted is determined by a variable window size
(context) which the function takes in as an input. Half the window size is
taken from left the word and half from the right of the word; thus window size
is the total number of words examined. For an odd window size, it rounds up
half the window size up for the part left of the word and rounds down for the
part right of the word.

\end{document}
